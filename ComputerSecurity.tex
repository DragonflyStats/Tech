Computer Security
2008 Paper
1) Access Control and Key Escrow
2) The RSA algorithm
3) Needham Schroeder
4) Hash Functions and Password Security Systems
    SHA - 1 Protocol
5) Digital Signature Schemes

Needham Schroeder Protocols
 
The term Needham–Schroeder protocol can refer to one of two communication protocols intended for use over an insecure network, both proposed by Roger Needham and Michael Schroeder.
 
These are:
The Needham–Schroeder Symmetric Key Protocol is based on a symmetric encryption algorithm. It forms the basis for the Kerberos protocol. This protocol aims to establish a session key between two parties on a network, typically to protect further communication.
The Needham–Schroeder Public-Key Protocol, based on public-key cryptography. This protocol is intended to provide mutual authentication between two parties communicating on a network, but in its proposed form is insecure.

 
The Secure Hash Algorithm
 
The Secure Hash Algorithm is one of a number of cryptographic hash functions published by the National Institute of Standards and Technology as a U.S. Federal Information Processing Standard.

SHA-1 Protocol
 
Chinese Remainder Theorem
 
A typical problem is to find integers x that simultaneously solve 
 
 
 
It's important in our applications that the two moduli be relatively prime; otherwise, we would have to check that the two congruences are consistent. The Chinese Remainder Theorem has a very simple answer:
 
Chinese Remainder Theorem: For relatively prime moduli m and n, the congruences
 



have a unique solution x modulo mn.
 
 
Our example problem would have a unique solution modulo 1627 =432.
It's better than this; there is a relatively simple algorithm to find the solution. Since all number theory algorithms ultimately come down to Euclid's algorithm, you can be sure it happens here as well.
First let's consider an even simpler example. Suppose we want all numbers x that satisfy
 

 
The numbers that satisfy the first congruence are in the sequence 

 


Just scan this sequence for a term that also leaves remainder 3 after division by 5. The answer is x=8.
 
Euclid's algorithm can be used to solve several problems: finding the greatest common divisor d of two numbers m and n, and finding two numbers x and y such that mx+ny=d, and solving congruences axbm  for x. We will now see how it also helps to solve Chinese Remainder problems.
