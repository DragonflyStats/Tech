	\documentclass[a4paper,12pt]{article}
%%%%%%%%%%%%%%%%%%%%%%%%%%%%%%%%%%%%%%%%%%%%%%%%%%%%%%%%%%%%%%%%%%%%%%%%%%%%%%%%%%%%%%%%%%%%%%%%%%%%%%%%%%%%%%%%%%%%%%%%%%%%%%%%%%%%%%%%%%%%%%%%%%%%%%%%%%%%%%%%%%%%%%%%%%%%%%%%%%%%%%%%%%%%%%%%%%%%%%%%%%%%%%%%%%%%%%%%%%%%%%%%%%%%%%%%%%%%%%%%%%%%%%%%%%%%
\usepackage{eurosym}
\usepackage{vmargin}
\usepackage{amsmath}
\usepackage{graphics}
\usepackage{epsfig}
\usepackage{subfigure}
\usepackage{enumerate}
\usepackage{fancyhdr}

\setcounter{MaxMatrixCols}{10}
%TCIDATA{OutputFilter=LATEX.DLL}
%TCIDATA{Version=5.00.0.2570}
%TCIDATA{<META NAME="SaveForMode"CONTENT="1">}
%TCIDATA{LastRevised=Wednesday, February 23, 201113:24:34}
%TCIDATA{<META NAME="GraphicsSave" CONTENT="32">}
%TCIDATA{Language=American English}

\pagestyle{fancy}
\setmarginsrb{20mm}{0mm}{20mm}{25mm}{12mm}{11mm}{0mm}{11mm}
\lhead{Apache Spark} \rhead{Kevin O'Brien} \chead{Spark Features} %\input{tcilatex}

\begin{document}
\section*{Spark Features}
%%--https://books.google.ie/books?isbn=3319292064

Spark takes MapReduce to the next level with less expensive shuffles in the data processing. With capabilities like in-memory data storage and near real-time processing, the performance can be several times faster than other big data technologies.

Spark also supports lazy evaluation of big data queries, which helps with optimization of the steps in data processing workflows. It provides a higher level API to improve developer productivity and a consistent architect model for big data solutions.

Spark holds intermediate results in memory rather than writing them to disk which is very useful especially when you need to work on the same dataset multiple times. It’s designed to be an execution engine that works both in-memory and on-disk. Spark operators perform external operations when data does not fit in memory. Spark can be used for processing datasets that larger than the aggregate memory in a cluster.

Spark will attempt to store as much as data in memory and then will spill to disk. It can store part of a data set in memory and the remaining data on the disk. You have to look at your data and use cases to assess the memory requirements. With this in-memory data storage, Spark comes with performance advantage.

Other Spark features include:
\begin{itemize}
\item  Supports more than just Map and Reduce functions.
\item  Optimizes arbitrary operator graphs.
\item  Lazy evaluation of big data queries which helps with the optimization of the overall data processing workflow.
\item  Provides concise and consistent APIs in Scala, Java and Python.
\item  Offers interactive shell for Scala and Python. This is not available in Java yet.
\end{itemize}


Spark is written in Scala Programming Language and runs on Java Virtual Machine (JVM) environment. It currently supports the following languages for developing applications using Spark:

\begin{itemize}
\item  Scala
\item Java
\item Python
\item Clojure
\item R
\end{itemize}
\end{document}
