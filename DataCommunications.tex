
CIS222 Data communications and

enterprise networking

 

Part 1

Section 1 – Basic Concepts

Section 2 – Network Architecture

Section 3 – The application Layer

Section 4 – The Transport Layer

Secton 5 – The Network Layer

Secton 6 – The Data Link Layer

Secton 7 – The Physical Layer

 

Section 1

Shannon’s Communication Model

Shannon’s law and Hamming codes.

DNS, Telnet protocols and Huffman coding.

The Spanning Tree problem

Dijkstra’s algorithm.

 

Mean Time Between Failures (MTBF)

MTBF is a basic measure of reliability for repairable items. It can be described as the number of hours that pass before a component, assembly, or system fails. It is a commonly-used variable in reliability and maintainability analyses.

MTBF can be calculated as the inverse of the failure rate for constant failure rate systems. For example: If a component has a failure rate of 2 failures per million hours, the MTBF would be the inverse of that failure rate.

MTBF = (1,000,000 hours) / (2 failures) = 500,000 hours

Mean Time To Failure (MTTF)

MTTF is a basic measure of reliability for non-repairable systems. It is the mean time expected until the first failure of a piece of equipment. MTTF is a statistical value and is meant to be the mean over a long period of time and large number of units. For constant failure rate systems, MTTF is the inverse of the failure rate. If failure rate is in failures/million hours, MTTF = 1,000,000 / Failure Rate for components with exponential distributions.

Technically MTBF should be used only in reference to repairable items, while MTTF should be used for non-repairable items. However, MTBF is commonly used for both repairable and non-repairable items.

 

