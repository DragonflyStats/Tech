Basic Statistics - spark.mllib
Summary statistics
Correlations
Stratified sampling
Hypothesis testing
Streaming Significance Testing
Random data generation
Kernel density estimation

Summary statistics
We provide column summary statistics for RDD[Vector] through the function colStats available in Statistics.

Scala
Java
Python
colStats() returns an instance of MultivariateStatisticalSummary, which contains the column-wise max, min, mean, variance, and number of nonzeros, as well as the total count.

Refer to the MultivariateStatisticalSummary Scala docs for details on the API.

import org.apache.spark.mllib.linalg.Vector
import org.apache.spark.mllib.stat.{MultivariateStatisticalSummary, Statistics}

val observations: RDD[Vector] = ... // an RDD of Vectors

// Compute column summary statistics.
val summary: MultivariateStatisticalSummary = Statistics.colStats(observations)
println(summary.mean) // a dense vector containing the mean value for each column
println(summary.variance) // column-wise variance
println(summary.numNonzeros) // number of nonzeros in each column
Correlations
Calculating the correlation between two series of data is a common operation in Statistics. In spark.mllib we provide the flexibility to calculate pairwise correlations among many series. The supported correlation methods are currently Pearson’s and Spearman’s correlation.

Scala
Java
Python
Statistics provides methods to calculate correlations between series. Depending on the type of input, two RDD[Double]s or an RDD[Vector], the output will be a Double or the correlation Matrix respectively.

Refer to the Statistics Scala docs for details on the API.

import org.apache.spark.SparkContext
import org.apache.spark.mllib.linalg._
import org.apache.spark.mllib.stat.Statistics

val sc: SparkContext = ...

val seriesX: RDD[Double] = ... // a series
val seriesY: RDD[Double] = ... // must have the same number of partitions and cardinality as seriesX

// compute the correlation using Pearson's method. Enter "spearman" for Spearman's method. If a 
// method is not specified, Pearson's method will be used by default. 
val correlation: Double = Statistics.corr(seriesX, seriesY, "pearson")

val data: RDD[Vector] = ... // note that each Vector is a row and not a column

// calculate the correlation matrix using Pearson's method. Use "spearman" for Spearman's method.
// If a method is not specified, Pearson's method will be used by default. 
val correlMatrix: Matrix = Statistics.corr(data, "pearson")
Stratified sampling
Unlike the other statistics functions, which reside in spark.mllib, stratified sampling methods, sampleByKey and sampleByKeyExact, can be performed on RDD’s of key-value pairs. For stratified sampling, the keys can be thought of as a label and the value as a specific attribute. For example the key can be man or woman, or document ids, and the respective values can be the list of ages of the people in the population or the list of words in the documents. The sampleByKey method will flip a coin to decide whether an observation will be sampled or not, therefore requires one pass over the data, and provides an expected sample size. sampleByKeyExact requires significant more resources than the per-stratum simple random sampling used in sampleByKey, but will provide the exact sampling size with 99.99% confidence. sampleByKeyExact is currently not supported in python.

Scala
Java
Python
sampleByKeyExact() allows users to sample exactly ⌈fk⋅nk⌉∀k∈K⌈fk⋅nk⌉∀k∈K items, where fkfk is the desired fraction for key kk, nknk is the number of key-value pairs for key kk, and KK is the set of keys. Sampling without replacement requires one additional pass over the RDD to guarantee sample size, whereas sampling with replacement requires two additional passes.

import org.apache.spark.SparkContext
import org.apache.spark.SparkContext._
import org.apache.spark.rdd.PairRDDFunctions

val sc: SparkContext = ...

val data = ... // an RDD[(K, V)] of any key value pairs
val fractions: Map[K, Double] = ... // specify the exact fraction desired from each key

// Get an exact sample from each stratum
val approxSample = data.sampleByKey(withReplacement = false, fractions)
val exactSample = data.sampleByKeyExact(withReplacement = false, fractions)
Hypothesis testing
Hypothesis testing is a powerful tool in statistics to determine whether a result is statistically significant, whether this result occurred by chance or not. spark.mllib currently supports Pearson’s chi-squared ( χ2χ2) tests for goodness of fit and independence. The input data types determine whether the goodness of fit or the independence test is conducted. The goodness of fit test requires an input type of Vector, whereas the independence test requires a Matrix as input.

spark.mllib also supports the input type RDD[LabeledPoint] to enable feature selection via chi-squared independence tests.

Scala
Java
Python
Statistics provides methods to run Pearson’s chi-squared tests. The following example demonstrates how to run and interpret hypothesis tests.

import org.apache.spark.SparkContext
import org.apache.spark.mllib.linalg._
import org.apache.spark.mllib.regression.LabeledPoint
import org.apache.spark.mllib.stat.Statistics._

val sc: SparkContext = ...

val vec: Vector = ... // a vector composed of the frequencies of events

// compute the goodness of fit. If a second vector to test against is not supplied as a parameter, 
// the test runs against a uniform distribution.  
val goodnessOfFitTestResult = Statistics.chiSqTest(vec)
println(goodnessOfFitTestResult) // summary of the test including the p-value, degrees of freedom, 
                                 // test statistic, the method used, and the null hypothesis.

val mat: Matrix = ... // a contingency matrix

// conduct Pearson's independence test on the input contingency matrix
val independenceTestResult = Statistics.chiSqTest(mat) 
println(independenceTestResult) // summary of the test including the p-value, degrees of freedom...

val obs: RDD[LabeledPoint] = ... // (feature, label) pairs.

// The contingency table is constructed from the raw (feature, label) pairs and used to conduct
// the independence test. Returns an array containing the ChiSquaredTestResult for every feature 
// against the label.
val featureTestResults: Array[ChiSqTestResult] = Statistics.chiSqTest(obs)
var i = 1
featureTestResults.foreach { result =>
    println(s"Column $i:\n$result")
    i += 1
} // summary of the test
Additionally, spark.mllib provides a 1-sample, 2-sided implementation of the Kolmogorov-Smirnov (KS) test for equality of probability distributions. By providing the name of a theoretical distribution (currently solely supported for the normal distribution) and its parameters, or a function to calculate the cumulative distribution according to a given theoretical distribution, the user can test the null hypothesis that their sample is drawn from that distribution. In the case that the user tests against the normal distribution (distName="norm"), but does not provide distribution parameters, the test initializes to the standard normal distribution and logs an appropriate message.

Scala
Java
Python
Statistics provides methods to run a 1-sample, 2-sided Kolmogorov-Smirnov test. The following example demonstrates how to run and interpret the hypothesis tests.

Refer to the Statistics Scala docs for details on the API.

import org.apache.spark.mllib.stat.Statistics

val data: RDD[Double] = ... // an RDD of sample data

// run a KS test for the sample versus a standard normal distribution
val testResult = Statistics.kolmogorovSmirnovTest(data, "norm", 0, 1)
println(testResult) // summary of the test including the p-value, test statistic,
                    // and null hypothesis
                    // if our p-value indicates significance, we can reject the null hypothesis

// perform a KS test using a cumulative distribution function of our making
val myCDF: Double => Double = ...
val testResult2 = Statistics.kolmogorovSmirnovTest(data, myCDF)
Streaming Significance Testing
spark.mllib provides online implementations of some tests to support use cases like A/B testing. These tests may be performed on a Spark Streaming DStream[(Boolean,Double)] where the first element of each tuple indicates control group (false) or treatment group (true) and the second element is the value of an observation.

Streaming significance testing supports the following parameters:

peacePeriod - The number of initial data points from the stream to ignore, used to mitigate novelty effects.
windowSize - The number of past batches to perform hypothesis testing over. Setting to 0 will perform cumulative processing using all prior batches.
Scala
StreamingTest provides streaming hypothesis testing.

val data = ssc.textFileStream(dataDir).map(line => line.split(",") match {
  case Array(label, value) => BinarySample(label.toBoolean, value.toDouble)
})

val streamingTest = new StreamingTest()
  .setPeacePeriod(0)
  .setWindowSize(0)
  .setTestMethod("welch")

val out = streamingTest.registerStream(data)
out.print()
Find full example code at "examples/src/main/scala/org/apache/spark/examples/mllib/StreamingTestExample.scala" in the Spark repo.
Random data generation
Random data generation is useful for randomized algorithms, prototyping, and performance testing. spark.mllib supports generating random RDDs with i.i.d. values drawn from a given distribution: uniform, standard normal, or Poisson.

Scala
Java
Python
RandomRDDs provides factory methods to generate random double RDDs or vector RDDs. The following example generates a random double RDD, whose values follows the standard normal distribution N(0, 1), and then map it to N(1, 4).

Refer to the RandomRDDs Scala docs for details on the API.

import org.apache.spark.SparkContext
import org.apache.spark.mllib.random.RandomRDDs._

val sc: SparkContext = ...

// Generate a random double RDD that contains 1 million i.i.d. values drawn from the
// standard normal distribution `N(0, 1)`, evenly distributed in 10 partitions.
val u = normalRDD(sc, 1000000L, 10)
// Apply a transform to get a random double RDD following `N(1, 4)`.
val v = u.map(x => 1.0 + 2.0 * x)
Kernel density estimation
Kernel density estimation is a technique useful for visualizing empirical probability distributions without requiring assumptions about the particular distribution that the observed samples are drawn from. It computes an estimate of the probability density function of a random variables, evaluated at a given set of points. It achieves this estimate by expressing the PDF of the empirical distribution at a particular point as the the mean of PDFs of normal distributions centered around each of the samples.

Scala
Java
Python
KernelDensity provides methods to compute kernel density estimates from an RDD of samples. The following example demonstrates how to do so.

Refer to the KernelDensity Scala docs for details on the API.

import org.apache.spark.mllib.stat.KernelDensity
import org.apache.spark.rdd.RDD

val data: RDD[Double] = ... // an RDD of sample data

// Construct the density estimator with the sample data and a standard deviation for the Gaussian
// kernels
val kd = new KernelDensity()
  .setSample(data)
  .setBandwidth(3.0)

// Find density estimates for the given values
val densities = kd.estimate(Array(-1.0, 2.0, 5.0))
