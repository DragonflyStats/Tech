Classification
Classification aims to divide items into categories. The most common classification type is binary classification, where there are two categories, usually named positive and negative. If there are more than two categories, it is called multiclass classification. spark.mllib supports two linear methods for classification: linear Support Vector Machines (SVMs) and logistic regression. Linear SVMs supports only binary classification, while logistic regression supports both binary and multiclass classification problems. For both methods, spark.mllib supports L1 and L2 regularized variants. The training data set is represented by an RDD of LabeledPoint in MLlib, where labels are class indices starting from zero: 0,1,2,…0,1,2,…. Note that, in the mathematical formulation in this guide, a binary label yy is denoted as either +1+1 (positive) or −1−1 (negative), which is convenient for the formulation. However, the negative label is represented by 00 in spark.mllib instead of −1−1, to be consistent with multiclass labeling.

Linear Support Vector Machines (SVMs)
The linear SVM is a standard method for large-scale classification tasks. It is a linear method as described above in equation (1)(1), with the loss function in the formulation given by the hinge loss:

L(w;x,y):=max{0,1−ywTx}.
L(w;x,y):=max{0,1−ywTx}.
By default, linear SVMs are trained with an L2 regularization. We also support alternative L1 regularization. In this case, the problem becomes a linear program.

The linear SVMs algorithm outputs an SVM model. Given a new data point, denoted by xx, the model makes predictions based on the value of wTxwTx. By the default, if wTx≥0wTx≥0 then the outcome is positive, and negative otherwise.

Examples

Scala
Java
Python
The following code snippet illustrates how to load a sample dataset, execute a training algorithm on this training data using a static method in the algorithm object, and make predictions with the resulting model to compute the training error.

Refer to the SVMWithSGD Scala docs and SVMModel Scala docs for details on the API.

import org.apache.spark.mllib.classification.{SVMModel, SVMWithSGD}
import org.apache.spark.mllib.evaluation.BinaryClassificationMetrics
import org.apache.spark.mllib.util.MLUtils

// Load training data in LIBSVM format.
val data = MLUtils.loadLibSVMFile(sc, "data/mllib/sample_libsvm_data.txt")

// Split data into training (60%) and test (40%).
val splits = data.randomSplit(Array(0.6, 0.4), seed = 11L)
val training = splits(0).cache()
val test = splits(1)

// Run training algorithm to build the model
val numIterations = 100
val model = SVMWithSGD.train(training, numIterations)

// Clear the default threshold.
model.clearThreshold()

// Compute raw scores on the test set.
val scoreAndLabels = test.map { point =>
  val score = model.predict(point.features)
  (score, point.label)
}

// Get evaluation metrics.
val metrics = new BinaryClassificationMetrics(scoreAndLabels)
val auROC = metrics.areaUnderROC()

println("Area under ROC = " + auROC)

// Save and load model
model.save(sc, "myModelPath")
val sameModel = SVMModel.load(sc, "myModelPath")
The SVMWithSGD.train() method by default performs L2 regularization with the regularization parameter set to 1.0. If we want to configure this algorithm, we can customize SVMWithSGD further by creating a new object directly and calling setter methods. All other spark.mllib algorithms support customization in this way as well. For example, the following code produces an L1 regularized variant of SVMs with regularization parameter set to 0.1, and runs the training algorithm for 200 iterations.

import org.apache.spark.mllib.optimization.L1Updater

val svmAlg = new SVMWithSGD()
svmAlg.optimizer.
  setNumIterations(200).
  setRegParam(0.1).
  setUpdater(new L1Updater)
val modelL1 = svmAlg.run(training)
Logistic regression
Logistic regression is widely used to predict a binary response. It is a linear method as described above in equation (1)(1), with the loss function in the formulation given by the logistic loss:
L(w;x,y):=log(1+exp(−ywTx)).
L(w;x,y):=log⁡(1+exp⁡(−ywTx)).

For binary classification problems, the algorithm outputs a binary logistic regression model. Given a new data point, denoted by xx, the model makes predictions by applying the logistic function
f(z)=11+e−z
f(z)=11+e−z
where z=wTxz=wTx. By default, if f(wTx)>0.5f(wTx)>0.5, the outcome is positive, or negative otherwise, though unlike linear SVMs, the raw output of the logistic regression model, f(z)f(z), has a probabilistic interpretation (i.e., the probability that xx is positive).

Binary logistic regression can be generalized into multinomial logistic regression to train and predict multiclass classification problems. For example, for KK possible outcomes, one of the outcomes can be chosen as a “pivot”, and the other K−1K−1 outcomes can be separately regressed against the pivot outcome. In spark.mllib, the first class 00 is chosen as the “pivot” class. See Section 4.4 of The Elements of Statistical Learning for references. Here is an detailed mathematical derivation.

For multiclass classification problems, the algorithm will output a multinomial logistic regression model, which contains K−1K−1 binary logistic regression models regressed against the first class. Given a new data points, K−1K−1 models will be run, and the class with largest probability will be chosen as the predicted class.

We implemented two algorithms to solve logistic regression: mini-batch gradient descent and L-BFGS. We recommend L-BFGS over mini-batch gradient descent for faster convergence.

Examples

Scala
Java
Python
The following code illustrates how to load a sample multiclass dataset, split it into train and test, and use LogisticRegressionWithLBFGS to fit a logistic regression model. Then the model is evaluated against the test dataset and saved to disk.

Refer to the LogisticRegressionWithLBFGS Scala docs and LogisticRegressionModel Scala docs for details on the API.

import org.apache.spark.SparkContext
import org.apache.spark.mllib.classification.{LogisticRegressionWithLBFGS, LogisticRegressionModel}
import org.apache.spark.mllib.evaluation.MulticlassMetrics
import org.apache.spark.mllib.regression.LabeledPoint
import org.apache.spark.mllib.linalg.Vectors
import org.apache.spark.mllib.util.MLUtils

// Load training data in LIBSVM format.
val data = MLUtils.loadLibSVMFile(sc, "data/mllib/sample_libsvm_data.txt")

// Split data into training (60%) and test (40%).
val splits = data.randomSplit(Array(0.6, 0.4), seed = 11L)
val training = splits(0).cache()
val test = splits(1)

// Run training algorithm to build the model
val model = new LogisticRegressionWithLBFGS()
  .setNumClasses(10)
  .run(training)

// Compute raw scores on the test set.
val predictionAndLabels = test.map { case LabeledPoint(label, features) =>
  val prediction = model.predict(features)
  (prediction, label)
}

// Get evaluation metrics.
val metrics = new MulticlassMetrics(predictionAndLabels)
val precision = metrics.precision
println("Precision = " + precision)

// Save and load model
model.save(sc, "myModelPath")
val sameModel = LogisticRegressionModel.load(sc, "myModelPath")
