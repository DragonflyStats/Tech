
Advanced Data Science Notepad

 


Artificial Neural Networks

Inductive Logic Programming (ILP)

Machine Learning

Naive Bayesian Classifier

Perceptrons

Quadratic Loss Function

Supervised learning

Training data sets

Winnow Algorithm


Artificial Neural Networks 

Artificial Neural Networks (ANNs) is an abstract simulation of a real nervous system that contains a collection of neuron units communicating with each other via axon connections. Such a model bears a strong reasemblance to axons and dendrites in a nervous system.


The first fundamental modeling of neural nets was proposeed in 1943 by McCulloch and Pitts in terms of a computational model of "nervous activity". The McCulloch-Pitts neuron is a binary device and each neuron has a fixed threshold logic. 


This model lead the works of John von Neumann, Marvin Minsky, Frank Rosenblatt, and many others. Hebb postulated, in his classical book The Organization of Behavior, that the neurons were appropiately interconnected by self-organization and that "an existing pathway strenghens the connections between the neurons". He proposed that the connectivity of the brain is continually changing as an organism learns 

different functional tasks, and that cells assemblies are created by such changes. By embedding a vast number of simple neurons in an interactive nervous system, it is possible to provide computational power  for very sophisticated informating processing. 


The neural model can be divided into two categories:


The first is the biological type. It encompasses networks mimicking biological neural systems such as audio functions or early vision functions.

The other type is application-driven. It depens less on the faithfulness to neurobiology. For this models the architectures are largely dictated by the application needs. Many such neural networks are represented by the so called connectionist models.

Inductive Logic Programming (ILP)

Inductive Logic Programming (ILP) is a research area formed at the intersection of Machine Learning and Logic Programming. ILP systems develop predicate descriptions from examples and background knowledge. The examples, background knowledge and final descriptions are all described as logic programs. A unifying theory of Inductive Logic Programming is being built up around lattice-based concepts such as refinement, least general generalisation, inverse resolution and most specific corrections. In addition to a well established tradition of learning-in-the-limit results, some results within Valiant's PAC-learning framework have been demonstrated for ILP systems. U-learnabilty, a new model of learnability, has also been developed.


Presently successful applications areas for ILP systems include the learning of structure-activity rules for drug design, finite-element mesh analysis design rules, primary-secondary prediction of protein structure and fault diagnosis rules for satellites.


The theory of ILP is based on proof theory and model theory for the first order predicate calculus. Inductive hypothesis formation is characterised by techniques including inverse resolution , relative least general generalisations , inverse implication , and inverse entailment.


The removal of redundancy, and use of search procedures also play an important role in the theory.


Computational Learning Theory (COLT) is used to analyse learning results for ILP systems. An extension to Valiant's PAC learning framework, U-learnability, has been suggested.

Machine Learning

Naive Bayesian Classifier

The naive Bayesian classifier is an algorithm for supervised learning that stores a single probabilistic summary for each class and that assumes conditional independence of the attributes given the class. Despite these simplifying assumptions, in many domains naive Bayes gives results as good or better than much more sophisticated approaches to induction. 

Perceptrons



Supervised learning


Supervised learning is the machine learning task of inferring a function from supervised training data. The training data consist of a set of training examples. In supervised learning, each example is a pair consisting of an input object (typically a vector) and a desired output value (also called the supervisory signal). A supervised learning algorithm analyzes the training data and produces an inferred function, which is called a classifier (if the output is discrete, see classification) or a regression function (if the output is continuous, see regression). The inferred function should predict the correct output value for any valid input object. This requires the learning algorithm to generalize from the training data to unseen situations 

in a "reasonable" way (see inductive bias). 

Training data sets

A training set is a set of data used in various areas of information science to discover potentially predictive relationships. Training sets are used in artificial intelligence, machine learning, genetic programming, intelligent systems, and statistics. In all these fields, a training set has much the same 

role and is often used in conjunction with a test set.


 

Winnow Algorithm

The Winnow algorithm



 


